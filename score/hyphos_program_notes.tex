\documentclass[11pt,letterpaper]{article}
\usepackage{fullpage}

\newcommand{\h}{{\it Hyphos}}

\begin{document}

\begin{quote}{{\it Text} means {\it Tissue;} but whereas hitherto we have always taken this tissue
as a product, a ready-made veil, behind which lies, more or less hidden, meaning (truth),
we are now emphasizing, in the tissue, the generative idea that the text is made, is 
worked out in a perpetual interweaving; lost in this tissue---this texture---the
subject unmakes himself, like a spider dissolving in the constructive secretions of its web.
Were we fond of neologisms, we might define the theory of the text as an {\it hyphology} 
({\it hyphos} is the tissue and the spider's web).} ---Roland Barthes, {\it The Pleasure of the Text}
\end{quote}

\h{} is a texture, a textile, consisting of many threads of {\it time} woven together
in a counterpoint of continuously fluctuating tempos. Although the musicians begin together, 
they quickly diverge from the tempo they share. This tempo, however, remains latent
throughout the piece becoming manifest for brief and fleeting moments when the threads
of the different musical parts
intertwine in just the right way to reveal its rhythms. 
The musicians may choose to use click-tracks to help them remain situated in time with respect to the 
other performers and computer, however, the piece may also be played without the aid
of click-tracks. In the latter case, the way in which the various musical threads
become entangled is, to a certain extent, variable and the resulting web gradually unmakes its creator.



%% The passage of time is subjective 
%% and subject to change, however, the act of playing music together (synchronizing) 
%% requires, at the very least, the illusion that the musicians have entrained themselves
%% to a temporal structure inherent in the piece of music they are performing. \h{},
%% as with a number of my recent works, is an attempt to explore

%in which case the ways in which the various musical threads become entangled
%is, to a certain extent, variable and the 


\end{document}
